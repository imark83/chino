% This is the file `arphic-sampler.tex' which shows the shapes
% of the Arphic fonts for simplified and traditional Chinese.
%
% written by Werner Lemberg <wl@gnu.org> 17-Jul-2005

\documentclass[13pt]{article}
\usepackage[margin=1.5cm]{geometry}
\usepackage{multicol}
\usepackage[T1]{CJKutf8}
\usepackage[overlap, CJK]{ruby}
\usepackage{pinyin}
\usepackage{setspace}
\renewcommand{\baselinestretch}{1.5}

\pagestyle{empty}


\begin{document}
\rubylatin
\renewcommand{\rubysep}{-1.2ex}
\renewcommand{\rubysize}{0.5}

\begin{CJK*}{UTF8}{}
\begin{multicols}{2}
% \CJKtilde
\LARGE
\CJKfamily{gkai}
\noindent
\ruby{买}{mǎi}\quad comprar\\
\ruby{吃}{chī}\quad comer\\
\ruby{炒}{chǎo}\quad saltear\\
\ruby{看}{kàn}\quad mirar, leer, ver\\
\ruby{学习}{xué xí}\quad estudiar\\
\ruby{洗澡}{xǐ zǎo}\quad ducharse\\
\ruby{睡觉}{shuì jiào}\quad dormir\\
\ruby{做}{zuò}\quad hacer\\
\ruby{回}{huí}\quad volver\\
\ruby{会}{huì}\quad saber\\
\ruby{听}{tīng}\quad escuchar\\
\ruby{土豆}{tǔ dòu}\quad patata\\
\ruby{胡萝卜}{hú luó bo}\quad zanahoria\\
\ruby{洋葱}{yáng cōng}\quad cebolla\\
\ruby{青椒}{qīng jiāo}\quad pimiento verde\\
\ruby{西红柿}{xī hóng shì}\quad tomate\\
\ruby{苹果}{píng guǒ}\quad manzana\\
\ruby{梨}{lí}\quad pera\\
\ruby{香蕉}{xiāng jiāo}\quad plátano\\
\ruby{橙子}{chéng zi}\quad naranja\\
\ruby{柑}{gān}\quad mandarina\\
\ruby{四季豆}{sì jì dòu}\quad judía verde\\
\ruby{菜花}{cài huā}\quad coliflor\\
\ruby{菠菜}{bō cài}\quad espinaca\\
\ruby{鱼}{yú}\quad pescado, pez\\
\ruby{猪肉}{zhū ròu}\quad carne de cerdo\\
\ruby{牛肉}{niú ròu}\quad carne de ternera\\
\ruby{鸡肉}{jī ròu}\quad carne de pollo\\
\ruby{羊肉}{yáng ròu}\quad carne de cordero\\


\end{multicols}
\end{CJK*}

\end{document}


%%% Local Variables:
%%% coding: utf-8
%%% mode: latex
%%% TeX-master: t
%%% End:
