\documentclass[13pt]{article}
\usepackage[margin=1.5cm]{geometry}
\usepackage{multicol}
\usepackage{CJKutf8}
\usepackage{xpinyin}
\usepackage[T1]{fontenc}
\usepackage{lmodern}

% \usepackage[overlap, CJK]{ruby}

\usepackage{setspace}
\renewcommand{\baselinestretch}{1.5}

\pagestyle{empty}


\begin{document}
% \rubylatin
% \renewcommand{\rubysep}{-1.2ex}
% \renewcommand{\rubysize}{0.5}

% \begin{CJK*}{UTF8}{gbsn}
\begin{CJK*}{UTF8}{gkai}
% \begin{multicols}{2}
\begin{pinyinscope}
\LARGE
\begin{enumerate}
	\item 除了\ldots 以外\ldots \qquad Aparte de\ldots también\ldots\\
	除了英语以外,我还会法语。\\
	Aparte de inglés, también hablo francés.
	\item 虽然\ldots 可是\ldots \qquad Aunque\\
	课虽然难, 可是很有意思。\\
	Aunque difíciles, las clases son interesantes.
	\item 因为\ldots 所以\ldots \qquad ya que\ldots entonces\ldots\\
	因为我病了,所以没有去学校。\\
	Ya que estaba enfermo entonces no he ido a clase.
	\item 不但\ldots 而且\ldots \qquad No solo\ldots sino también\ldots\\
	这件衣服不但好看,而且价格也很\xpinyin{便}{pian2}\xpinyin{宜}{yi5}。\\
	Esta prenda no solo es bonita, si no que también es barata.
	\item 是\ldots 还是\ldots  \qquad o\ldots (preguntar)\\
	这双袜子是姐\xpinyin{姐}{jie5}的,还是弟\xpinyin{弟}{di5}的?\\
	¿Estos calcetines son de tu hermana o de tu hermano?
	\item 或者\ldots  \qquad o\ldots (asertivo)\\
	今天我想在家听音\xpinyin{乐}{yue4}或者看电影。\\
	Hoy me gustaría escuchar música en casa o ver una película.
	\item 一\ldots 就\ldots  \qquad tan pronto como\ldots\\
	我一到家,就给你打电话。\\
	Tan pronto como llegue a casa te llamaré por teléfono.
	\item 要是\ldots 就\ldots  \qquad si\ldots entonces\ldots\\
	要是今天出太阳,我们就去公园。\\
	Si hoy sale el sol, entocnes iremos al parque.
	\item 不是\ldots 就是\ldots  \qquad Si no\ldots entoces\ldots (elección)\\
	今天不是星期一,就是星期二。\\
	Si hoy no es lunes, entonces es martes.
	\item 不然\ldots  \qquad si no\ldots (consecuencia)\\
	便宜一点儿吧,不然我不买。\\
	Abarátalo un poco, si no, no lo compro.
\end{enumerate}

\end{pinyinscope}
% \end{multicols}
\end{CJK*}

\end{document}
